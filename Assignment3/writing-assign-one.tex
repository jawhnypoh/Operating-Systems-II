\documentclass[onecolumn, draftclsnofoot,10pt, compsoc]{IEEEtran}
\usepackage{graphicx}
\usepackage{url}
\usepackage{setspace}

\usepackage{geometry}
\geometry{margin = 0.75in}

\title{Writing Assignment One}
\author{Andrew Davis\\CS 444 Oregon State University\\Spring 2018}
\date {17 April 2018}

%%%%%%%%%%%%%%%%%%%%%%%%%%%%%%%%%%%%%%%
\begin{document}
\begin{titlepage}
    \pagenumbering{gobble}
	\maketitle
    \begin{singlespace}
        \begin{abstract}
        % 6. Fill in your abstract    
		This assignment will cover the basics of process and thread management, as well as CPU scheduling in the Windows operating system (OS), and FreeBSD.
		These two operating systems will be compared to Linux, looking at differences, similiarities, and reasons for these comparisons to exist.
        	
%This document is written using one sentence per line.
%This allows you to have sensible diffs when you use \LaTeX with version control, as well as giving a quick visual test to see if sentences are too short/long.
%If you have questions, ``The Not So Short Guide to LaTeX'' is a great resource (\url{https://tobi.oetiker.ch/lshort/lshort.pdf})
        \end{abstract}     
    \end{singlespace}
\end{titlepage}
\newpage
\pagenumbering{arabic}

\clearpage

% 8. now you write!
\section{Windows Overview}
	Windows operating system (OS) is a, "priority-drive, preemptive scheduling system," \cite{windowswebsite}.
	This means that the system assigns priority to threads (not processes) based on importance.
	Once these threads are given a priority number, the highest priority is ran, while those that are less important wait their turn.
	Threads are given a set amount of time to run, at which a new thread is ran when that time is met.
	At any point, if a thread of higher priority comes about, the old thread (lower priority) is cut off and the new one (higher priority) begins to run.\\

	A restriction on these high priority threads is the processor it is running on, creating a condition known as \textit{processor affinity}.
	Here, the thread can be bound, or unbound, to a specific CPU or processor, meaning a thread with high priority may not be allowed to run on a certain processor.\\

	The scheduler for Windows OS takes place in the kernel \cite{windowswebsite}. 
	Rather than a scheduler function, the kernel dispatches threads based on priority, as stated above.
	There are three reasons that may require a thread to be dispatched \cite{windowswebsite}:
	\begin{enumerate}
		\item a new thread has become readily available,
		\item a thread exits its run state due to its time being up, it is interrupted, it enters a wait state, etc.,
		\item the priority of a thread changes - is lower than a new thread, or
		\item a thread must run on a different processor
	\end{enumerate}
	Once any of these four things happen, the OS performs something called a context switch on the thread affected. 
	This saves data from the previously running thread, and switches to a new one. \\

	A large part of Windows is its concentration on threads. 
	As a result, resources are not devoted to figuring out which threads belong to which processes.
	An example is outlined in \cite{windowswebsite};
	say there are two processes \textit A and \textit B. 
	If \textit A has 5 threads, and \textit\ B has 2 threads, then the CPU devotes \(\frac{1}{7}\)th of the time to each one, as opposed to 50\% of processing power to \textit A and 50\% of processing power to \textit B.\\

	In conclusion, it is important to note that the key of process management in Windows is the focus on threads. The importance of this will be covered when comparing/contrasting Windows to Linux.
	
\section{FreeBSD Overview}
	FreeBSD is a Unix based OS that is related to Linux. The way in which it manages processes is very similar, if not a mirror image on Linux. Due to the similarities, the overview of FreeBSD will be short. \\

	FreeBSD is a split up into two levels of schedulers: low-level and a high level \cite{bsdschedule}.
	The first, low-level scheduler, runs everytime an existing thread is blocked and a new one must be ran.
	The kernel organizes run queues that place threads in order from low- to high-priority \cite{bsdschedule}.
	When this happens, the low-level scheduler's job is to select the thread with the highest priority and run it.
	The high-level scheduler takes runnable threads and assigns them a level of priority.
	Once this is one, the thread is placed in the run queue and responsibility is handed off to the low-level scheduler. \\ 

	As far as processes and threads go, Linux and FreeBSD are similar.

\section{Comparison}
	The largest differences between the three operating systems lie between Windows and Linux.
	This is largely due to the fact that Windows is not Unix based, like Linux and FreeBSD.
	Some notable differences between Windows and Linux are:
	\begin{itemize}
		\item closing a parent process in Linux kills all children processes, whereas the same action in Windows allows the children to stay alive \cite{windowdiff}, and
		\item the only shared information between parent and child processes (threads in the case of Windows), is the parent process id, whereas Linux shares much more \cite{windowdiff}.
	\end{itemize} 

	While there are large differences in the management of process between the two systems, the CPU scheduling shares some similiarities, such as:
	\begin{itemize}
		\item priority in processes/threads are used to calculate execution, and
		\item both systems have processor affinity.
	\end{itemize}

	As far as Linux and FreeBSD go, the differences are near non-existant for process/thread management.
	This is due to, as stated earlier, their branching from Unix.

%Bibliography referenced from: https://www.sharelatex.com/learn/Bibliography_management_with_bibtex
\newpage
\begin{thebibliography}{10}
	\bibitem{windowswebsite}
	Mark E. Russinovich, David A. Solomon.
	Processes, Threads, and Jobs in the Windows Operating Systems.
	June 17 2009.
	\\\texttt{https://www.microsoftpressstore.com/articles/article.aspx?p=2233328\&seqNum=7}

	\bibitem{bsdschedule}
	Marshall K. McKusick, George V. Neville-Neil, Robert N.M. Watson.
	Process Management in the FreeBSD Operating System.
	October 2 2014.
	\\\textit{www.informit.com/articles/article.aspx?p=2249436\&seqNum=4}

	\bibitem{windowdiff}
	Shankar Raman.
	Process bahavior in Windows and Linux.
	September 22 2014.
	\\\texttt{https://shankaraman.wordpress.com/tag/differences-between-windows-and-linux-process}




















\end{thebibliography}
\end{document}